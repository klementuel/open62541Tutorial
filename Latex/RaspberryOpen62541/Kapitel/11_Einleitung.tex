\section{Einleitung} \label{sec:einleitung}
Diese Technische Dokumentation dient als Anleitung zur Einrichtung eines OPC UA Servers auf einem Raspberry Pi4. Als Basis dient hierbei die Open Source-Variante von OPC UA open62541. Um das Verständnis dieser Dokumentation zu erhöhen, wird empfohlen, die dazugehörige Bachelor-Arbeit zu lesen. Dort werden sehr viele Informationen aus den Übungen von \url{https://opcua.rocks/} verarbeitet. Diese Seite bietet ausführliche Informationen zu open62541 und den Arbeiten mit offiziellen Informationsmodellen der OPC UA Fundation
Als Grundvoraussetzung wird folgendes gesetzt:
\begin{itemize}
	\item Raspberry Pi4 mit 2GB Ram
	\item Unified Automation UA Expert \\
	(\url{www.unified-automation.com/products/development-tools.html})
	\item Unified Automation UA Modeler \\
	(\url{www.unified-automation.com/products/development-tools.html})
	\item FreeOpcUA Modeler (\url{github.com/FreeOpcUa/opcua-modeler})
	\item Raspberry Pi Imager (\url{www.raspberrypi.org/software/})
	\item Für Windows Putty als SSH Client
	(\url{https://www.putty.org/})
	\item Cyberduck, um Zugriff via SSH auf Datei-System zu bekommen\\
	(\url{https://cyberduck.io/download/}) 
\end{itemize}

